
\section{Introducción}%
\label{sec:introduccion}

\subsection{Resumen}%
\label{sub:resumen}

El presente trabajo expondrá el desarrollo de un proyecto a realizarse en la Universidad Nacional Arturo Jauretche (UNAJ), que
consistirá en la elaboración de un sistema informático para el registro de datos\@. El mismo tendrá como finalidad la consolidación
de conocimientos adoptados en la carrera y requerirá de la capacidad de análisis e investigación para adaptarse a los estándares
y herramientas utilizadas por la Dirección de Informática de la institución.

\subsection{Objetivos}%
\label{sub:objetivos}


El objetivo general de este proyecto es el desarrollo de un sistema que permita el registro de datos sobre las actividades
realizadas por los integrantes de la UNAJ\@. El mismo almacenará datos de proyectos de investigación, publicaciones, voluntariados, etc.

Actualmente la universidad no cuenta con una plataforma establecida para el registro de esta información, sólo se dispone de
los datos de las personas que integran la institución. Estos datos están distribuidos en dos sistemas:
\begin{itemize}
    \item \textbf{SIU-Mapuche:} Sistema de RRHH de la Universidad, cuenta con toda la información relevante al personal (docente
        y no-docente) y a los cargos.
    \item \textbf{SIU-Guaraní:} Sistema Académico de la Universidad, aquí es donde se encuentran los datos de alumnos, institutos
        , materias, comisiones, carreras, etc.
\end{itemize}

El uso de estos sistemas, como fuente de información y su integración con la plataforma RUDA, permitirá saber en qué actividades
está o estuvo involucrada determinada persona en la UNAJ. Se desarrollará una interfaz web para el acceso a los datos
, para lo que será necesario relacionar la información en los sistemas fuente con la de RUDA\@. Además, se implementará
un servicio REST, que permitirá acceder fácilmente a los datos.

Este proyecto está pensado para ser utilizado por las siguientes áreas:
\begin{itemize}
    \item Centro de Política Educativa
    \item Centro de Política y Territorio
    \item Secretaría Económico Financiera
\end{itemize}


\subsubsection{Objetivos específicos}%
\label{ssub:objetivos_especificos}

Para llevar adelante el mencionado proyecto, se pretenden alcanzar los siguientes objetivos específicos:

\begin{itemize}
    \item \textbf{Modelar la estructura de los datos a almacenar en el sistema.}
    \item \textbf{Desarrollar una iterfaz de usuario para la interacción con los datos:} aquí se presentará la información de cada persona y sus actividades.
    \item \textbf{Implementar la integración de los sistemas:} se deberá trabajar en un método para obtener los datos externos e incorporarlos al sistema RUDA.
    \item \textbf{Desarrollar una API REST para la obtención de los datos:} se desarrollará un servicio web que facilite la
        manipulación de datos a través de internet. Esto brindará a la universidad la capacidad de reutilizar esta fuente de datos en futuros proyectos.
\end{itemize}
