\item \textbf{Asset:} administra la generación de URL y el versionado de hojas de estilo css, archivos JavaScript e imágenes.
\item \textbf{Console:} permite crear comandos para usar en consola. Bastante útil para tareas recurrentes.
\item \textbf{Dotenv:} administra las variables de entorno de la aplicación.
\item \textbf{Expression Language:} permite utilizar expresiones dentro de archivos de configuración para obtener lógica más compleja.
\item \label{itm:flex} \textbf{Flex:} es un componente que facilita la integración de paquetes de terceros a través de lo que se denomina recetas Symfony. Estas recetas consisten en un conjunto de instrucciones automatizadas.
\item \textbf{Form:} permite crear, procesar y reutilizar formularios.
\item \textbf{Framework:} define la configuración principal del framework.
\item \textbf{Monologbundle:} integra la librería monolog con symfony para el registro de mensajes.
\item \textbf{ORM Pack:} es la librería encargada del mapeo objeto-relacional.
\item \textbf{Process:} librería utilizada para la ejecución de subprocesos. Resuelve problemas relacionados con la diferencia entre sistemas operativos y además provee una ejecución segura.
\item \textbf{Security:} componente de seguridad de Symfony. Se encarga de definir el control de acceso, sistemas de autenticación y además de establecer los proveedores de usuarios.
\item \textbf{Serializer:} paquete de Symfony que se encarga de transformar un objeto en un formato adecuado para la transmisión de datos. Ej: JSON.
\item \textbf{Swiftmailer:} permite el envío de emails a través de un servidor propio o de terceros.
\item \textbf{Translation:} permite definir textos en la aplicación para traducción al idioma local del usuario. Este proceso es comúnmente llamado internacionalización.
\item \textbf{Twig:} Motor de templates preferido por Symfony, permite renderizar contenido html de manera fácil, organizada y segura.
\item \textbf{Validator:} componente encargado de la validación de datos. Weblink: incrementa la performance de la aplicación al utilizar HTTP2 y funciones de precarga en navegadores modernos.
\item \textbf{Yaml:} se encarga de convertir archivos YAML en arrays PHP. Gran parte de la configuración de Symfony se encuentra definida en este formato.
