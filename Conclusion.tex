
\section{Conclusión}
\label{sec:conclusion}
El hecho de no contar con un sistema que almacene la información de cada
actividad extracurricular realizada por las personas de la UNAJ presenta
un problema para la institución y sus integrantes\@. Estos datos
no solo son valiosos debido a propósitos administrativos, cada una de
estas actividades le otorga un valor agregado a la historia académica del
estudiante y deberían ser registradas al igual que que sus
materias y calificaciones.


Con esto en mente, se consiguió implementar una interfaz web que ayudará a la administración
de estos datos. Será posible solicitar la información de una determinada
persona y obtener una lista sus actividades extracurriculares y cargos\@.
Cada actividad contará con una interfaz administrativa, lo que
permitirá agregar personas al sistema y asignarle las actividades o cargos
que se requiera\@. Además, se terminó por definir un servicio REST que
obtiene parte de su información del sistema \textbf{Mapuche} e integra
los datos de ambos sistemas\@.


Con respecto al sistema \textbf{Guaraní}, no se alcanzó a implementar la
integración por razones de tiempo, se debe ajustar el sistema RUDA para
que los dos tipos de datos de personas coexistan entre si: una persona puede
estar trabajando en la universidad, es decir, registrada en el sistema
\textbf{Mapuche} y al mismo tiempo estudiando. Estos dos datos deben
vincularse para representar una sola persona y además, indicar estos tipos
de situaciones.

Contar con esta información puede dar lugar a análisis de datos,
estadística e incluso el desarrollo de otros proyectos. Será posible,
por ejemplo, identificar alumnos que sean muy activos en áreas de
investigación: lo que podría dar lugar a un diferente tipo de orientación
o tutoría para un mejor aprovechamiento de sus habilidades\@. Además de esto, quizás
en un futuro se pueda listar esta información en un comprobante o anexo al titulo universitario
de manera de resaltar los méritos de la persona.

Mejorar los procesos de administración de la información de la Universidad es muy necesario
y útil a futuro, ya que beneficia a todos los que la integran.

\section{Reflexión}%
\label{sec:reflexión}

Aprender un \textit{framework} no resulta ser un proceso trivial, requiere de aprender los
distintos componentes que lo conforman y en parte, su funcionamiento interno\@. Es necesario
utilizar y leer código con el que no se está familiarizado y que pasó por muchas revisiones
hasta llegar al público.

Trabajar con código de terceros es una buena manera de ejercitar la habilidad de \textit{leer}
código\@. Esta habilidad puede abrir la puerta a contribuciones a proyectos open source o
personales.

Durante el transcurso de esta PPS, no sólo aprendí a trabajar con un nuevo \textit{framework},
también me familiaricé con diferentes flujos de trabajo y herramientas\@. Definitivamente es
el proyecto más grande que formé parte y por consiguiente, el que más aprendizaje me otorgó.

El proyecto en sí me presentó un desafío por el hecho de tener que sobrellevar el desarrollo
del sistema con la escritura y organización del documento\@. Escribir un documento académico
puede volverse una tarea bastante desorganizada (al menos en mi caso) utilizando procesadores
de texto comunes, esto es por que ni bien crece el documento, crecen la cantidad de secciones
que se deben modificar y mantener\@. Esto es lo que me llevó a aprender el sistema {\LaTeX}\@.
El mismo solucionó gran parte de mis problemas de organización, al poder
separar el documento en archivos individuales para cada sección\@. Además, la inserción de
gráficos o secciones de código están manejadas internamente y no hay necesidad de
preocuparse por los
epígrafes, ya que son enumerados automáticamente\@. Como si esto fuera poco {\LaTeX} permite
definir funciones: esto permite programar diferentes tipos de contenido reutilizable en todo el
documento.


Por otro lado, este proyecto me llevó a descubrir el área en la cual quiero desenvolverme como
futuro Ingeniero, me interesa bastante la optimización y/o automatización de procesos de todo
tipo. Quizás también relacionado con la programación o administración de sistemas\@. Previo a
este proyecto no tenía una idea bastante formada respecto a este tema, así que me fue de gran
ayuda.
