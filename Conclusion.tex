
\section{Conclusión}
\label{sec:conclusion}
El hecho de no contar con un sistema que almacene la información de cada
actividad extracurricular realizada por las personas de la UNAJ presenta
un problema para la institución y sus integrantes\@. Estos datos
no solo son valiosos debido a propósitos administrativos, cada una de
estas actividades le otorga un valor agregado a la historia académica del
estudiante y deberían ser registradas al igual que que sus
materias y calificaciones.


Se consiguió implementar una interfaz web que ayudará a la administración
de estos datos. Será posible solicitar la información de una determinada
persona y obtener una lista sus actividades extracurriculares y cargos\@.
Cada actividad contará con una interfaz administrativa, lo que
permitirá agregar personas al sistema y asignarle las actividades o cargos
que se requiera\@. Además, se terminó por definir un servicio REST que
obtiene parte de su información del sistema \textbf{Mapuche} e integra
los datos de ambos sistemas.



Contar con esta información puede dar lugar a análisis de datos,
estadística e incluso el desarrollo de otros proyectos. Será posible,
por ejemplo, identificar alumnos que sean muy activos en áreas de
investigación: lo que podría dar lugar a un diferente tipo de orientación
o tutoría para un mejor aprovechamiento de sus habilidades\@.


