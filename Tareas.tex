
\subsection{Tareas}%
\label{sub:tareas}
A continuación, se detallarán las tareas por ejecutar en vistas al cumplimiento de los objetivos previamente definidos:

\subsubsection{Modelado conceptual de datos}%
\label{ssub:modelado_conceptual_de_datos}
Se definirá el modelado de datos a implementar basándose en los datos que se requieren almacenar y como están relacionados.

\subsubsection{Creación de la base de datos}
\label{ssub:creacioDB}

Entidades a desarrollar que almacenarán datos  sobre la duración de cada actividad:

\begin{itemize}
    \item Cargos de Autoridad
\item Consejeros Superiores
\item Miembros y Roles en las Comisiones del Consejo Superior
\item Asambleístas
\item Consejeros Consultivos
\item Responsables de Áreas
\item Directores/Subdirectores de Institutos
\item Directores/Subdirectores de Carreras
\item Proyectos de Investigación
\item Proyectos de Extensión
\item Programas
\item Actividades de Divulgación
\item Publicaciones
\item Cursos de Extensión
\item Voluntariados
\item Vinculadores
\item Programas
\item Becas
\item Pasantías
\item Movilidad RTF
\item Movilidad Conurbano Sur
\item Prácticas Profesionales Supervisadas

\end{itemize}

\subsubsection{Definición de ABMs}%
\label{ssub:definición_de_abms}


Una ABM permite al usuario interactuar con los datos, por esto es que se deberá definir cada una en particular:\newline

Relacionadas a la política de la institución:
\begin{itemize}
        \item Rector
    \item Vicerrector
    \item Asambleísta
    \item Consejero Superior
    \item Responsable de Área
    \item Coordinador de Materia
    \item Miembro de Consejo Consultivo
    \item Director de Instituto
    \item Subdirector de Instituto
    \item Miembro de Comisión de Consejo Superior
    \item Director de Carrera
    \item Subdirector de Carrera
    \item Instituto
    \item Materia
    \item Carrera
    \item Consejo Consultivo
    \item Área
    \item Resolución Administrativa
    \item Comisión de Consejo Superior
    \item Rol de Comisión de Consejo Superior

\end{itemize}
\ \\No relacionadas a la política de la institución:
\begin{itemize}
        \item Práctica Profesional Supervisada
    \item Participación en Práctica Profesional Supervisada
    \item Actividad de Divulgación
    \item Curso de Extensión
    \item Participación en Curso de Extensión
    \item Pasantía
    \item Participación en Pasantía
    \item Programa
    \item Miembro de Programa
    \item Movilidad
    \item Participación en Movilidad
    \item Actividades de Vinculación
    \item Becas
    \item Participación en Beca
    \item Voluntariado
    \item Participación en Voluntariado
    \item Proyecto
    \item Miembro de Proyecto
    \item Rol de Proyecto

\end{itemize}

\subsubsection{Desarrollo de API }%
\label{ssub:desarrollo_de_api_}
Para este desarrollo se generarán rutas mediante las cuales se podrá acceder a la misma información detallada en la tarea anterior.

\begin{itemize}
    \item \textbf{Implementación de nodos para obtener cada dato:} cada dato debe ser accesible mediante una petición http.
    \item \textbf{Implementación de autenticación del servicio}
\end{itemize}

\subsubsection{Instalación y configuración de librerías}%
\label{ssub:instalación_y_configuración_de_librerias}
Se deberá configurar cada librería de acuerdo a las necesidades del sistema.
A continuación se listan las librerías a utilizar para el desarrollo:

\begin{itemize}
    \item Sonata-admin
\item Sonata-user
\item FOSUser
\item Doctrine
\item Data-Fixtures
\item Faker
\item Api-Platform

\end{itemize}
